
\documentclass[format=acmsmall, review=false, screen=true]{acmart}

\usepackage{booktabs} % For formal tables

\usepackage[ruled]{algorithm2e} % For algorithms
\renewcommand{\algorithmcfname}{ALGORITHM}
\SetAlFnt{\small}
\SetAlCapFnt{\small}
\SetAlCapNameFnt{\small}
\SetAlCapHSkip{0pt}
\IncMargin{-\parindent}

% Copyright
\setcopyright{acmlicensed}

% Document starts
\begin{document}
% Title portion. Note the short title for running heads 
\title[Short Title]{Effects of Solid State Drives on Database Performance}  
\author{Philip M. Westrich}
\affiliation{
  \institution{Tennessee Technological University}
  \streetaddress{1 William L. Jones Drive}
  \city{Cookeville} \state{TN} \postcode{38501} \country{USA}
}

\begin{abstract}

    In the last decade or so, a new form of flash-based persistent storage, known as solid state drives (SSDs), have begun 
    to replace traditional mechanical hard disk drives (HDDs). They are orders of magnitude faster, lighter, more energy 
    efficient, and less prone to sudden impact than HDDs. As their prices fall and storage capacities increase, they have 
    taken more of a hold in both the consumer and enterprise market. These new SSDs behave differently than HDDs, which can 
    have a significant effect on any read or write heavy workload, such as database management systems. In this paper, we explore 
    what these effects are and their proposed solutions.
 
\end{abstract}

\maketitle

\section{Introduction}

Text goes here.

\section{Overview of SSDs}

Text goes here.

\section{Effects on database systems}

Text goes here.

\section{Solutions to these effects}

Text goes here.

\section{Conclusions}

Text goes here.

\bibliographystyle{ACM-Reference-Format}
\bibliography{bibliography}

\end{document}
