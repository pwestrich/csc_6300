
\documentclass[format=acmsmall, review=false, screen=true]{acmart}

\usepackage{booktabs} % For formal tables

\usepackage[ruled]{algorithm2e} % For algorithms
\renewcommand{\algorithmcfname}{ALGORITHM}
\SetAlFnt{\small}
\SetAlCapFnt{\small}
\SetAlCapNameFnt{\small}
\SetAlCapHSkip{0pt}
\IncMargin{-\parindent}

% Copyright
\setcopyright{acmlicensed}

% Document starts
\begin{document}
% Title portion. Note the short title for running heads 
\title[Short Title]{Effects of Solid State Drives on Database Performance}  
\author{Philip M. Westrich}
\affiliation{
  \institution{Tennessee Technological University}
  \streetaddress{1 William L. Jones Drive}
  \city{Cookeville} \state{TN} \postcode{38501} \country{USA}
}

\begin{abstract}

    In the last decade or so, a new form of flash-based persistent storage, known as solid state drives (SSDs), have begun 
    to replace traditional mechanical hard disk drives (HDDs). They are orders of magnitude faster, lighter, more energy 
    efficient, and less prone to sudden impact than HDDs. As their prices fall and storage capacities increase, they have 
    taken more of a hold in both the consumer and enterprise market. These new SSDs behave differently than HDDs, which can 
    have a significant effect on any read or write heavy workload, such as database management systems. In this paper, we explore 
    what these effects are and their proposed solutions.
 
\end{abstract}

\maketitle

\section{Introduction}

For years, the gap in performance between processors and persistent storage has continually widened. Disk latency for 
traditional hard disk drives (HDDs) has improved by approximately ten percent per year, while we are just now beginning 
to see the end of Moore's Law for processors. \cite{Xie2011}

Traditional HDDs rely on mechanical parts, and therefore are destined to fail eventually, take relative eons to read 
or write data, and are prone to multiuple methods of failure, such as sudden force, vibration, or simply mechanical 
failure. They also are power inefficent and generate pleny of heat. \cite{Xie2011}

With the introduction of solid state drives (SSDs), we can begin to address many of the aforementioned issues, as they 
are not susceptible to the same problems. SSDs are lighter, faster, more energy efficient, and resistant to blunt force.
\cite{Xie2011} 

However, they are not completely free from issue. At the time of writing, SSDs are still much more expensive per gigabyte 
than their HDD counterparts, though they promise to reach equivalence within a few years. They also have much different 
performance characteristics than HDDs, most notably, usually being worse at small random writes. They also tend to become 
slower as they fill up, and have a limited number of read/write/erase cycles they can endure before failing. 
\cite{Xie2011, Dumitru2007}

Overall, SSDs are better than their predecessor, and will most likely come to replace them in the coming years. Therefore, 
applications that are heavily reliant on their storage medium, such as databases, must come prepared for it. In this paper, 
we will explore what effects these performance differences have as well as the solutions that have been proposed.

\section{Overview of solid state drives}

Though solid state drives are designed to emulate the behavior of previous block devices, under the hood, they operate 
quite differently. \cite{Lee2008, Cornwell2012, Micheloni2013, MatejFucek2014} In this section, we explain the general 
architecture of a solid state drive and explain these differences.

Solid state drives consist of three major components: an array of non-volatile flash memory, a host interface, and a 
microcontroller that bridges the gap between the two. 

\subsection{NAND flash}

Most SSDs today use a type of electrically erasable programmable read-only memory (EEPROM) known as NAND flash memory. 
NAND is an abbreviation for 'not-and', a type of logic gate. These NAND cells are arranged into a grid pattern. These 
cells can only be accessed in certain patterns dependent on the particular drive's arrangement of those cells. 
\cite{Cornwell2012, Micheloni2013}

Several of these grids, usually one to four, are placed together into what is known as a die. Each die can read somewhere 
around 400 MB/sec, while they can only write at around 20MB/sec. This is due to the much more complicated write method 
they employ. However, they have much lower latencies for these operations, typically measured in microseconds, and in 
order to meet capacity and speed requirements, multiple dies are packaged inside one SSD. \cite{Cornwell2012, Micheloni2013}

SSDs use a write-verify method to program, or write to, the cells. In this procedure, the cells are first erased, then 
a high voltage signal is applied to them until they reach the appropriate value. As the cells hold more bits, this 
process gets more difficult and slows down, as the voltages applied to the cells must become more fine-grained and precise.
\cite{Cornwell2012, Micheloni2013}

\subsection{SSD controller}

The controller's job is twofold. First, it must translate the operations requested by the host machine into operations 
that be performed on the NAND flash. Second, it must do this in a manner such that wear on the device is minimized and 
even, and the time needed to perform the requested operations is minimized. 

To carry out these tasks, SSDs have a small computer, known as a microcontroller. The microcontroller is also paired with 
specialized hardware to speed up its computations. 

\section{Effects on database systems}

Text goes here.

\section{Solutions to these effects}

Text goes here.

\section{Conclusions}

Text goes here.

\bibliographystyle{ACM-Reference-Format}
\bibliography{bibliography}

\end{document}
